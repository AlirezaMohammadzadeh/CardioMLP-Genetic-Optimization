\documentclass[12pt,a4paper]{book}
\usepackage{geometry}
\usepackage{graphicx}
\usepackage{hyperref}
\usepackage{amsmath}
\usepackage{amssymb}
\usepackage{booktabs}
\usepackage{multirow}
\usepackage{array}
\usepackage{float}
\usepackage{parskip}
\usepackage[utf8]{inputenc}
\usepackage[english]{babel}
\usepackage{csquotes}
\usepackage{xepersian}

\settextfont{XB Zar}
\setdigitfont{Yas}

\input{LatexTemplate/commands}
\input{LatexTemplate/dicen2fa}

\geometry{a4paper, margin=1in}
\hypersetup{
    colorlinks=true,
    linkcolor=blue,
    filecolor=magenta,      
    urlcolor=cyan,
}

\title{بررسی و مقایسه الگوریتم‌های یادگیری ماشین برای پیش‌بینی بیماری‌های قلبی عروقی}
\author{نام نویسنده}
\date{\today}

\begin{document}

\frontmatter
\maketitle
\include{LatexTemplate/Chant}

\begin{abstract}
این تحقیق به بررسی و مقایسه الگوریتم‌های مختلف یادگیری ماشین و یادگیری عمیق برای پیش‌بینی بیماری‌های قلبی عروقی (CVD) می‌پردازد. با استفاده از دیتاست حاوی اطلاعات ۷۰٬۰۰۰ فرد و عوامل خطر مختلف، مدل‌های مختلفی شامل درخت تصمیم، جنگل تصادفی، KNN، SVM، Gaussian Naive Bayes، رگرسیون لجستیک و شبکه‌های عصبی عمیق مورد ارزیابی قرار گرفته‌اند. نتایج نشان می‌دهد که الگوریتم‌های یادگیری عمیق به دلیل توانایی در یادگیری الگوهای پیچیده، دقت بالاتری در پیش‌بینی CVD نشان می‌دهند.
\end{abstract}

\tableofcontents

\mainmatter
\include{LatexTemplate/chapter1}  % مقدمه
\include{LatexTemplate/chapter2}  % درباره دیتاست
\include{LatexTemplate/chapter3}  % انتخاب ویژگی
\include{LatexTemplate/chapter4}  % الگوریتم‌های کلاسیک
\include{LatexTemplate/chapter5}  % یادگیری عمیق
\include{LatexTemplate/chapter6}  % نتایج و نتیجه‌گیری

\backmatter
\section*{تقدیر و تشکر}
این تحقیق با حمایت مالی ... انجام شده است.

\bibliographystyle{plain}
\bibliography{references}
\begin{thebibliography}{9}
\bibitem{cvd-paper} مقاله CVD\_paper
\bibitem{kaggle-dataset} دیتاست عوامل خطر CVD، \\\url{https://data.world/kudem}
\end{thebibliography}

\end{document} 